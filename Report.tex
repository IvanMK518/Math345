%%%%%%%%%%%%%%%%%%%%%%%%%%%%%%%%%%%%%%%%%
% University Assignment Title Page 
% LaTeX Template
% Version 1.0 (27/12/12)
%
% This template has been downloaded from:
% http://www.LaTeXTemplates.com
%
% Original author:
% WikiBooks (http://en.wikibooks.org/wiki/LaTeX/Title_Creation)
%
% License:
% CC BY-NC-SA 3.0 (http://creativecommons.org/licenses/by-nc-sa/3.0/)
%%%%%%%%%%%%%%%%%%%%%%%%%%%%%%%%%%%%%%%%%

%----------------------------------------------------------------------------------------
%	PACKAGES AND OTHER DOCUMENT CONFIGURATIONS
%----------------------------------------------------------------------------------------

\documentclass[12pt]{article}
\usepackage[english]{babel}
\usepackage[utf8x]{inputenc}
\usepackage{amsmath,amssymb,amsthm}
\usepackage{amsfonts,amsthm,xcolor,color}
\usepackage{graphicx}
\usepackage[colorinlistoftodos]{todonotes}

% Uncomment the line below if you have the trackchanges package installed
%\usepackage[inline]{./trackchanges-0.7.0/LatexPackage/trackchanges}

% Setup commands for 6 editors (uncomment if using trackchanges)
%\addeditor{GM1}
%\addeditor{GM2}
%\addeditor{GM3}
%\addeditor{GM4}
%\addeditor{GM5}

% Define custom commands
\newcommand{\HRule}{\rule{\linewidth}{0.5mm}}
\newcommand{\Deriv}[3]{\dfrac{\partial^#1 #2}{\partial #3^#1}}

\begin{document}

\begin{titlepage}

\center % Center everything on the page
 
%----------------------------------------------------------------------------------------
%	HEADING SECTIONS
%----------------------------------------------------------------------------------------

\textsc{\LARGE EMORY UNIVERSITY}\\[0.3cm]
\textsc{\LARGE DEPARTMENT OF MATHEMATICS}\\[0.3cm]
\textsc{\Large Atlanta (GA) USA}\\[0.3cm]
\textsc{\Large Project 1}\\[0.5cm]

%----------------------------------------------------------------------------------------
%	TITLE SECTION
%----------------------------------------------------------------------------------------

\HRule \\[0.4cm]
{\large \bfseries MATH345 MATH MODELING}\\[0.03cm]
\HRule \\[1.5cm]

%----------------------------------------------------------------------------------------
%	AUTHOR SECTION
%----------------------------------------------------------------------------------------

\begin{minipage}{0.4\textwidth}
\begin{flushleft} \large
\emph{Submitted By:} \\ Ivan Martinez-Kay
Group Name: \\ Mississipi
\end{flushleft}
\end{minipage}
~
\begin{minipage}{0.4\textwidth}
\begin{flushright} 
\small \emph{Authors:}\\[0.3cm]
\textsc{Ivan Martinez-Kay}\\[0.3cm]
\textsc{Daniel Chen}\\[0.3cm]
\textsc{Charlotte Zhao}\\[0.3cm]
\end{flushright}
\end{minipage}\\[1cm]

%----------------------------------------------------------------------------------------
%	DATE SECTION
%----------------------------------------------------------------------------------------

{\Large Fall 2025 \\Group Report 1 \\ \today}\\[1cm]

%----------------------------------------------------------------------------------------
%	LOGO SECTION
%----------------------------------------------------------------------------------------

% Uncomment and use your university logo
%\includegraphics[height=2cm,width=7cm]{EU_hz_280.png}\\[1cm]
 
%----------------------------------------------------------------------------------------

\vfill

\end{titlepage}

\newpage
\sffamily
\pagenumbering{gobble}

\section*{Assignment}

\noindent
\textsc{Text of the Project}

\newpage
\pagecolor{white}
\rmfamily
\setcounter{page}{1}
\pagenumbering{arabic}

\begin{abstract}
EXECUTIVE SUMMARY (MAX 1 PAGE)

\vspace{1cm}
\noindent
CONTRIBUTION STATEMENT: 

\noindent
All the authors equally participated in the project.
\textsf{or specify the different contributions, reporting possible disagreements.}
\end{abstract}

\section{Introduction}

10 page limit (excluding the bibliography, including the executive summary)

Your introduction goes here! 

Some examples of commonly used commands and features are listed below, to help you get started.

If you have a question, please e-mail me.

\section{Some \LaTeX{} Examples}
\label{sec:examples}

\subsection{Sections}

Use section and subsection commands to organize your document. \LaTeX{} handles all the formatting and numbering automatically. Use ref and label commands for cross-references.

\subsection{Comments}

Comments can be added to the margins of the document using the \todo{Here's a comment in the margin!} todo command, as shown in the example on the right. You can also add inline comments too:

\todo[inline, color=green!40]{This is an inline comment.}

\subsection{Tables and Figures}

Use the table and tabular commands for basic tables --- see Table~\ref{tab:widgets}, for example. You can upload a figure (JPEG, PNG or PDF) using the files menu. To include it in your document, use the includegraphics command as in the code for Figure~\ref{fig:frog} below.

% Commands to include a figure:
\begin{figure}
\centering
% Uncomment and use your image file
%\includegraphics[width=0.5\textwidth]{frog.jpg}
\caption{\label{fig:frog}This is a figure caption.}
\end{figure}

\begin{table}
\centering
\begin{tabular}{l|r}
Item & Quantity \\\hline
Widgets & 42 \\
Gadgets & 13
\end{tabular}
\caption{\label{tab:widgets}An example table.}
\end{table}

\subsection{Mathematics}

\LaTeX{} is great at typesetting mathematics. Let $X_1, X_2, \ldots, X_n$ be a sequence of independent and identically distributed random variables with $\text{E}[X_i] = \mu$ and $\text{Var}[X_i] = \sigma^2 < \infty$, and let

\begin{equation}\label{eq:1}
S_n = \frac{X_1 + X_2 + \cdots + X_n}{n}
      = \frac{1}{n}\sum_{i}^{n} X_i
\end{equation}

This is a derivative:
$$
\Deriv{2}{u}{x}
$$

denote their mean. Then as $n$ in equation \eqref{eq:1} approaches infinity, the random variables $\sqrt{n}(S_n - \mu)$ converge in distribution to a normal $\mathcal{N}(0, \sigma^2)$.

\subsection{Lists}

You can make lists with automatic numbering \dots

\begin{enumerate}
\item Like this,
\item and like this.
\end{enumerate}
\dots or bullet points \dots
\begin{itemize}
\item Like this,
\item and like this.
\end{itemize}

We hope you find write\LaTeX\ useful, and please let us know if you have any feedback using the help menu above.

\subsection{Working with an offline trackchange}

This template includes the library {\tt trackchanges} to perform offline trackchanges.

A list of possible editors can be added at the preamble. For now, each Group Member is denoted by GMn where n is a number. Feel free to edit with your initials. This will help track who made the comments.

% Uncomment the examples below if using trackchanges:
% This is a sentence but \add[GM1]{GM thinks it should be more complete}.
% This is another sentence but GM2 thinks that the word \change[GM2]{sentence}{phrase} is wrong.
% This a third sentence, but GM5 wants to remove the word \remove[GM5]{sentence}.  
% GM2 thinks that this is all insane\note[GM4]{He is probably right}.
% GM3 says that a reference is needed \refneeded[GM3]{}.

To do a rapid analysis of all the comments, there is a python script that opens a GUI for processing the text and accept/reject/modify the comments. Look at the directory {\tt trackchanges-0.7.0}. 

Otherwise, you can do this manually.

\section{How to include References}

In this system, you write a file with extension .bib and then you include references with the command \texttt{cite}. Example: \cite{example2025}. The count of pages (max. 10) does not include the References.

\newpage
\pagenumbering{gobble}
\bibliographystyle{plain}
\bibliography{References}

\end{document}